\chapter{Final Remarks}
\label{finalremarks}

In this report, we described the activities concluded in the first six months of the doctorate program. The focus of our thesis proposal is to provide new studies about the needs of second language learners, present more reliable datasets for various tasks of the LS pipeline, and propose novel solutions to the limitations of modern LS approaches.

We have presented an unfinished version of a literature survey, which outlines in detail the state-of-the-art of each step of the LS pipeline and discusses the merits and limitations of each strategy. We have presented the results of two experiments with the tasks of Substitution Generation and Selection. We have found that, although word alignment alone is not enough to provide valid substitutions for complex words, selecting an adequate tool for such task can considerably influence the quality of substitutions produced. We have also discovered that our modeling of the Substitution Selection task as a classification problem is not feasible, since it would require for training instances to be created for each word deemed complex by a certain target audience.

Finally, we have provided a detailed plan of the activities to be conducted throughout the doctorate program. By the end of the first year, we expect to have concluded at least one of the user studies which we intend to conduct, and also to be able to present results on our experiments with spoken text corpora in the task of Substitution Ranking. And by the end of the doctorate program, we expect to provide new information about the needs of second language learners with respect to LS, new datasets for various tasks of the LS pipeline, and also a reliable LS strategy for the English language.

In the future, we intend to explore Lexical Simplification strategies that go beyond replacing complex words and expressions for simpler equivalents, such as removing unimportant information from sentences and learning deep simplification rules from parallel corpora by combining constituency and dependency parses.