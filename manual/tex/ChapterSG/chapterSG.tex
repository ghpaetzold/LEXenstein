\chapter{The Substitution Generation Module}
\label{sg}

We define Substitution Generation (SG) as the task of producing candidate substitutions for complex words. Authors commonly address this task by querying WordNet \cite{wordnet} and UMLS\cite{Bodenreider04}. Some examples of authors who resort to this strategy are \cite{Devlin1998} and \cite{Carroll99}. Recently however, learning substitutions from aligned corpora have become a more popular strategy \cite{Paetzold2013} and \cite{Horn2014}.

In the Substitution Generation module of LEXenstein (lexenstein.generators), one has access to several Substitution Generation methods available in literature. The module contains a series of classes, each representing a distinct approach in literature. Currently, it offers support for $5$ distinct approaches. All approaches use LEXenstein's Text Adorning module, described in Section~\ref{adorning}, to create substitutions for all possible inflections of verbs and nouns. The following Sections describe each one individually.








\section{PaetzoldGenerator}

Employs a novel strategy, in which substitutions are extracted from tagged word embedding models.

\subsection{Parameters}

To be instantiated, this class requires as input a path to a binary tagged word vector model trained with word$2$vec. The model can be trained over a POS tag annotated corpus using any of the parameters supported by word$2$vec. To learn more on how to produce the model, please refer to Chapter~\ref{chapterresources}.



\subsection{Example}

The code snippet below shows the PaetzoldGenerator class being used:

\begin{lstlisting}
from lexenstein.generators import *
from lexenstein.spelling import *

nc = NorvigCorrector('spelling_model.bin', format='bin')

pg = PaetzoldGenerator('tagged_embeddings_model.bin', nc, './pos_model.tagger', './stanford-postagger.jar', '/usr/bin/java')
subs = pg.getSubstitutions('lexmturk.txt', 10)
\end{lstlisting}











\section{GlavasGenerator}

Employs the strategy described in \cite{glavas2015}, in which substitutions are extracted from typical word embedding models.

\subsection{Parameters}

To be instantiated, this class requires as input a path to a binary word vector model trained with word$2$vec. The model can be trained over any type of corpus using any of the parameters supported by word$2$vec. To learn more on how to produce the model, please refer to Chapter~\ref{chapterresources}.



\subsection{Example}

The code snippet below shows the GlavasGenerator class being used:

\begin{lstlisting}
from lexenstein.generators import *

kg = GlavasGenerator('embeddings_model.bin')
subs = kg.getSubstitutions('lexmturk.txt', 10)
\end{lstlisting}













\section{KauchakGenerator}

Employs the strategy described in \cite{Horn2014}, in which substitutions are automatically extracted from parallel corpora.

\subsection{Parameters}

To be instantiated, this class requires as input an object of the MorphAdornerToolkit class, a parsed document of parallel sentences, the word alignments between them in Pharaoh format, a list of stop words and a NorvigCorrector object. To learn more on how to produce the model, please refer to Chapter~\ref{chapterresources}.

\subsection{Example}

The code snippet below shows the KauchakGenerator class being used:

\begin{lstlisting}
from lexenstein.morphadorner import MorphAdornerToolkit
from lexenstein.generators import *
from lexenstein.spelling import *

nc = NorvigCorrector('spelling_model.bin', format='bin')

m = MorphAdornerToolkit('./morph/')

kg = KauchakGenerator(m, 'parallel.txt', 'alignments.txt', 'stop_words.txt', nc)
subs = kg.getSubstitutions('lexmturk.txt')
\end{lstlisting}

































\section{YamamotoGenerator}

Employs the strategy described in \cite{Yamamoto2013}, in which substitutions are extracted from dictionary definitions of complex words.

\subsection{Parameters}

This approach requires as input an API key for the Merriam Dictionary\footnote{http://www.dictionaryapi.com/}, which can be obtained for free, and a NorvigCorrector object. To get the key, follow the steps below:

\begin{enumerate}
\item Visit the page \url{http://www.dictionaryapi.com/register/index.htm}.
\item Fill in your personal information.
\item In "Request API Key \#1:" and "Request API Key \#2:", select "Collegiate Dictionary" and "Collegiate Thesaurus".
\item Login in \url{http://www.dictionaryapi.com}.
\item Visit your "My Keys" page.
\item Use the "Dictionary" key to create a YamamotoGenerator object.
\end{enumerate}

\subsection{Example}

The code snippet below shows the YamamotoGenerator class being used:

\begin{lstlisting}
from lexenstein.morphadorner import MorphAdornerToolkit
from lexenstein.generators import *
from lexenstein.spelling import *

nc = NorvigCorrector('spelling_model.bin', format='bin')

m = MorphAdornerToolkit('./morph/')

yg = YamamotoGenerator(m, '0000-0000-0000-0000', nc)
subs = yg.getSubstitutions('lexmturk.txt')
\end{lstlisting}













\section{MerriamGenerator}

Extracts a dictionary linking words to their synonyms, as listed in the Merriam Thesaurus.

\subsection{Parameters}

This approach requires as input an API key for the Merriam Thesaurus\footnote{http://www.dictionaryapi.com/}, which can be obtained for free, and a NorvigCorrector object. To get the key, follow the steps below:

\begin{enumerate}
\item Visit the page \url{http://www.dictionaryapi.com/register/index.htm}.
\item Fill in your personal information.
\item In "Request API Key \#1:" and "Request API Key \#2:", select "Collegiate Dictionary" and "Collegiate Thesaurus".
\item Login in \url{http://www.dictionaryapi.com}.
\item Visit your "My Keys" page.
\item Use the "Thesaurus" key to create a MerriamGenerator object.
\end{enumerate}

\subsection{Example}

The code snippet below shows the MerriamGenerator class being used:

\begin{lstlisting}
from lexenstein.morphadorner import MorphAdornerToolkit
from lexenstein.generators import *
from lexenstein.spelling import *

nc = NorvigCorrector('spelling_model.bin', format='bin')

m = MorphAdornerToolkit('./morph/')

mg = MerriamGenerator(m, '0000-0000-0000-0000', nc)
subs = mg.getSubstitutions('lexmturk.txt')
\end{lstlisting}















\section{WordnetGenerator}

Extracts a dictionary linking words to their synonyms, as listed in WordNet. 

\subsection{Parameters}

It requires for a NorvigCorrector object, the path to a POS tagging model, and the path to the Stanford Tagger. To learn more on how to obtain these resources, please refer to Chapter~\ref{chapterresources}.

\subsection{Example}

The code snippet below shows the WordnetGenerator class being used:

\begin{lstlisting}
from lexenstein.morphadorner import MorphAdornerToolkit
from lexenstein.generators import *
from lexenstein.spelling import *

nc = NorvigCorrector('spelling_model.bin', format='bin')

m = MorphAdornerToolkit('./morph/')

wg = WordnetGenerator(m, nc, './pos_model.tagger', './stanford-postagger.jar', '/usr/bin/java')
subs = wg.getSubstitutions('lexmturk.txt')
\end{lstlisting}









\section{BiranGenerator}

Employs the strategy described in \cite{Biran2011}, in which substitutions are filtered from the Cartesian product between vocabularies of complex and simple words.

\subsection{Parameters}

This approach requires as input vocabularies of complex and simple words, as well as two Language Models trained over complex and simple corpora, a NorvigCorrector object, the path to a POS tagging model, and the path to the Stanford Tagger. To learn more on how to obtain these resources, please refer to Chapter~\ref{chapterresources}.

\subsection{Example}

The code snippet below shows the BiranGenerator class being used:

\begin{lstlisting}
from lexenstein.morphadorner import MorphAdornerToolkit
from lexenstein.generators import *
from lexenstein.spelling import *

nc = NorvigCorrector('spelling_model.bin', format='bin')

m = MorphAdornerToolkit('./morph/')

bg = BiranGenerator(m, 'vocab_complex.txt', 'vocab_simple.txt', 'lm_complex.bin', 'lm_simple.bin', nc, './pos_model.tagger', './stanford-postagger.jar', '/usr/bin/java')
subs = bg.getSubstitutions('lexmturk.txt')
\end{lstlisting}