\chapter{The Complex Word Identification Module}
\label{cwi}

We define Complex Word Identification (CWI) as the task of deciding which words can be considered complex by a certain target audience. There are not many examples in literature of approaches for the task. In \cite{Shardlow2013cwi} the performance of three approaches are compared over the dataset introduced by \cite{cwcorpus}. The LS strategy of \cite{Horn2014} provides an implicit approach for the task.

In the Complex Word Identification module of LEXenstein (lexenstein.identifiers), one has access to several Complex Word Identification methods. The module contains a series of classes, each representing a distinct method. Currently, it offers support for $5$ distinct approaches. The following Sections describe each one individually.

\section{MachineLearningIdentifier}

Uses Machine Learning techniques to train classifiers that distinguish between complex and simple words. The class allows for the user to train models with $4$ Machine Learning techniques: Support Vector Machines, Decision Trees and linear models estimated with Stochastic Gradient Descent and Passive Aggressive Learning. As input, it requires for a FeatureEstimator object. As output, it produces a vector containing a binary label for the target word of each of the $N$ instances in a dataset in VICTOR or CWICTOR format. The label will have value $1$ if the target word was predicted as complex, and $0$ otherwise.

\subsection{Parameters}

During instantiation, the MachineLearningIdentifier requires only for a FeatureEstimator object. The user can then use the ``calculateTrainingFeatures'' and ``calculateTestingFeatures'' functions to estimate features of training and testing datasets in VICTOR or CWICTOR formats, the ``selectKBestFeatures'' function to apply feature selection over the features estimated, the ``trainSVM'', ``trainDecisionTreeClassifier'', ``trainSGDClassifier'' and ``trainPassiveAggressiveClassifier'' functions to train CWI models, and finally the ``identifyComplexWords'' function to produce output labels. To know more about the parameters and recommended values of the aforementioned functions, please refer to LEXenstein's documentation.

\subsection{Example}

The code snippet below shows the MachineLearningIdentifier class being used:

\begin{lstlisting}
from lexenstein.identifiers import *
from lexenstein.features import *

fe = FeatureEstimator()
fe.addLexiconFeature('lexicon.txt', 'Simplicity')
fe.addLengthFeature('Complexity')
fe.addSenseCountFeature('Simplicity')

mli = MachineLearningIdentifier(fe)
mli.calculateTrainingFeatures('train_cwictor_corpus.txt')
mli.calculateTestingFeatures('test_victor_corpus.txt')
mli.selectKBestFeatures(k=5)
mli.trainDecisionTreeClassifier()

labels = mli.identifyComplexWords()
\end{lstlisting}





\section{LexiconIdentifier}

Uses a lexicon of complex/simple words in order to judge the complexity of a word: if it appears in the lexicon, then it is complex/simple. As input, it requires for a lexicon file of complex or simple words and expressions. As output, it produces a vector containing a binary label for the target word of each of the $N$ instances in a dataset in VICTOR or CWICTOR format. The label will have value $1$ if the lexicon classifies the word as complex, and $0$ otherwise.

\subsection{Parameters}

During instantiation, the LexiconIdentifier requires for a lexicon file and an indicator label. The lexicon file must be in plain text and contain one word/expression per line, and the value of the label must be either ``complex'' or ``simple''. If the label's value is ``complex'', then the lexicon will be interpreted as a vocabulary of complex words and expressions, otherwise it will be interpreted as a vocabulary of simple words.

\subsection{Example}

The code snippet below shows the LexiconIdentifier class being used:

\begin{lstlisting}
from lexenstein.identifiers import *

li = LexiconIdentifier('lexicon.txt', 'simple')

labels = li.identifyComplexWords('test_victor_corpus.txt')
\end{lstlisting}



\section{ThresholdIdentifier}

Estimates the threshold $t$ over a given feature value that best separates complex and simple words. As input, it requires for a FeatureEstimator object. As output, it produces a vector containing a binary label for the target word of each of the $N$ instances in a dataset in VICTOR or CWICTOR format. The label will have value $1$ if the lexicon classifies the word as complex, and $0$ otherwise.

\subsection{Parameters}

During instantiation, the ThresholdIdentifier requires for a FeatureEstimator object. The user can then use the ``calculateTrainingFeatures'' and ``calculateTestingFeatures'' functions to estimate features of training and testing datasets in VICTOR or CWICTOR formats, the ``trainIdentifierBruteForce'' and ``trainIdentifierBinarySearch'' functions to train CWI models, and finally the ``identifyComplexWords'' function to produce output labels.

The ``trainIdentifierBruteForce'' and ``trainIdentifierBinarySearch'' functions require for a feature index as input. It will determine over which feature of the ones included in the FeatureEstimator object the threshold $t$ will be estimated. To know more about the parameters and recommended values of the aforementioned functions, please refer to LEXenstein's documentation.

\subsection{Example}

The code snippet below shows the ThresholdIdentifier class being used:

\begin{lstlisting}
from lexenstein.identifiers import *
from lexenstein.features import *

fe = FeatureEstimator()
fe.addLengthFeature('Complexity')
fe.addSenseCountFeature('Simplicity')

ti = ThresholdIdentifier(fe)
ti.calculateTrainingFeatures('train_cwictor_corpus.txt')
ti.calculateTestingFeatures('test_victor_corpus.txt')
ti.trainIdentifierBruteForce(1)

labels = ti.identifyComplexWords()
\end{lstlisting}